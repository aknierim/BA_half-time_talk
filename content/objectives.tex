% \subsection{Scientific Objectives}
% ===============================================================================================================================
\begin{frame}{Gamma-ray bursts}
  \begin{itemize}
    \item Most luminous explosions after the Big Bang
    \item Defined by a prompt emission phase and rapid, irregular variability
    \item Followed by decaying afterglow phase
    \item $T_{90}\sim\numrange{0.01}{1000}\si{\second}$, divided into two populations:
    \begin{itemize}
      \item [\to] Short, hard GRBs: $T_{90} \lesssim \SI{2}{\second}$
      \item [\to] Long, soft GRBs:  $T_{90} \gtrsim \SI{2}{\second}$
    \end{itemize}
    \item Likely to originate from ultra-relativistic jets, triggered by certain types of stellar core collapse events
  \end{itemize}
\end{frame}

% ===============================================================================================================================
\begin{frame}{Galactic transients}
  \begin{itemize}
    \item Pulsar wind nebula (PWN) flares:
    \begin{itemize}
      \item [\to] Bubbles of relativistic plasma, energised by winds from NSs
    \end{itemize}
    \item Magnetar giant flares:
    \begin{itemize}
      \item [\to] NSs powered by their very high magnetic fields
    \end{itemize}
    \item Microquasars:
    \begin{itemize}
      \item [\to] NSs or BHs, activated by accreting matter from their companion stars, generating collimated jets of plasma
    \end{itemize}
    \item Transient binary pulsars:
    \begin{itemize}
      \item [\to] Some transient X-ray binary pulsars may contain pulsar winds interacting with material from their stellar companions
    \end{itemize}
    \item Novae:
    \begin{itemize}
      \item [\to] Thermonuclear explosions due to matter accretion onto white dwarfs
    \end{itemize}
    \item Unidentified HE transients
  \end{itemize}
\end{frame}

% ===============================================================================================================================
\begin{frame}{X-ray, optical and radio transients}
  \begin{itemize}
    \item Tidal disruption events (TDEs)
    \begin{itemize}
      \item [\to] Occurs when a star aproaches a SMBH and is pulled apart by the BHs tidal force
      \item [\to] Sufficiently massive galaxies are expected to harbour such SMBHs in their central regions
    \end{itemize}
    \item Supernova shock breakout (SSB)
    \begin{itemize}
      \item [\to] Early UV to X-ray flashes on timescales of hours
      \item [\to] Emission powered by radioactive decay on timescales of weeks
    \end{itemize}
    \item Fast radio bursts (FRBs)
    \begin{itemize}
      \item [\to] Radio pulse on timescales of milliseconds
      \item [\to] Physical process yet to be understood
    \end{itemize}
  \end{itemize}
\end{frame}

% ===============================================================================================================================
\begin{frame}{HE neutrino transients}
  \begin{itemize}
    \item Follow-up of neutrino alerts may lead to origin of hadronic cosmic-ray sources
    \item Possible connection to fainter GRBs or flaring states of persistent objects
    \begin{itemize}
      \item [\to] Use CTA to study spatial and temporal coincidence between neutrinos and gamma rays
    \end{itemize}
  \end{itemize}
\end{frame}

% ===============================================================================================================================
\begin{frame}{GW transients}
  \begin{itemize}
    \item Binary NS and BH mergers predicted as strongest sources of GWs
    \item Direct detection would help probe dynamical behaviour of relativistic compact objects
    \item NS-NS or NS-BH mergers are candidates for short GRBs
  \end{itemize}
\end{frame}

% ===============================================================================================================================
\begin{frame}{Serendipitous VHE transients}
  \begin{itemize}
    \item Their unpredictable nature makes them hard to detect
    \item Large FoV and high instantaneous sensitivity of CTA as key advantages
    \item Fast acting RTA system
  \end{itemize}
\end{frame}

% ===============================================================================================================================
\begin{frame}{VHE transient survey}
  \begin{itemize}
    \item Divergent pointing mode for simultaneous coverage of large FoV
    \begin{itemize}
      \item [\to] Entails some loss in energy resolution and angular resolution
    \end{itemize}
    \item Detection of GRBs from their onset
    \item Unbiased searches for VHE transients
  \end{itemize}
\end{frame}
