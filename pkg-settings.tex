% package settings
\unimathsetup{
  math-style=ISO,
  bold-style=ISO,
  sans-style=italic,
  nabla=upright,
  partial=upright,
  mathrm=sym,
}

\sisetup{
  separate-uncertainty=true,
  per-mode=reciprocal,
  output-decimal-marker={.},
  range-phrase = \text{--},
}
\DeclareSIUnit\crab{Crab}

% tikz settings
\usetikzlibrary{overlay-beamer-styles,calc,tikzmark,decorations.pathreplacing}
\tikzset{fontscale/.style = {font=\relsize{#1}}}

\setmathfont{XITS Math}[range={scr, bfscr}]
\setmathfont{XITS Math}[range={cal, bfcal}, StylisticSet=1]


% bibliography settings
\begin{filecontents*}{\jobname.bib}
@mastersthesis{hackfeld,
  author      = {Hackfeld, J. and Nöthe, M.},
  title       = {Analyzing the Data Volume Reduction for the LST-1 Prototype of the Cherenkov Telescope Array},
  year        = {2021},
  address     = {Bochum}
}
@online{perezdiaz,
  author       = {Pérez Diaz, G.},
  year         = {2016},
  url          = {https://www.cta-observatory.org/about/how-cta-works/},
  organization = {CTA/~IAC},
  urldate      = {2022-07-10}
}
\end{filecontents*}
% reference source
\addbibresource{\jobname.bib}

% remove footnote marks
\makeatletter
\def\@makefnmark{}
\makeatletter

\setbeamertemplate{footnote}{%
  \parindent 1em\noindent
  \raggedright
  \insertfootnotetext\par
}

% This adds a circle with a picture of your choice in it.
% Usage:
% \roundpic[<optional arguments>]{<radius of the cirlce [cm]>}{<picture width [cm]>}{<path_to_picture>}{x pos}{y pos}
\newcommand{\roundpic}[6][]{%
  \node [circle, draw, color=tugreen, minimum width = #2,
    path picture = {
      \node [#1] at (path picture bounding box.center) {
        \includegraphics[width=#3]{#4}};
    }] at (#5,#6) {};}%

% Comment this out to represent vectors with an arrow on top.
% Uncomment this to represent vectors as bold symbols.
\renewcommand{\vec}[1]{\mathbf{#1}}


\newcommand{\code}[2]{%
  \texttt{\textcolor{#1}{\detokenize{#2}}}%
}